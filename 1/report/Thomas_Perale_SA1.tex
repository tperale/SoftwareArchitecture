
\documentclass[a4paper,11pt]{article}

\usepackage[english]{babel}
\usepackage[utf8x]{inputenc}
\usepackage{amsmath}
\usepackage{graphicx}
\usepackage{cite}
\usepackage{hyperref}
\usepackage{fullpage}
\usepackage{listings}
\usepackage{xcolor}

\definecolor{dkgreen}{rgb}{0,0.6,0}
\definecolor{gray}{rgb}{0.5,0.5,0.5}
\definecolor{mauve}{rgb}{0.58,0,0.82}

\lstdefinestyle{scala}{
  frame=tb,
  language=scala,
  aboveskip=3mm,
  belowskip=3mm,
  showstringspaces=false,
  columns=flexible,
  basicstyle={\small\ttfamily},
  numbers=none,
  numberstyle=\tiny\color{gray},
  keywordstyle=\color{blue},
  commentstyle=\color{dkgreen},
  stringstyle=\color{mauve},
  frame=single,
  breaklines=true,
  breakatwhitespace=true,
  tabsize=3,
}

\setlength{\parskip}{1.2ex}
\begin{document}

\title{}
\title{Software Architecture\\Assignment 1: DDD Onion Architectural Pattern}
\author{Thomas Perale -- 0546990}

\maketitle

The goal of this assignment was to replace the \emph{Cake Pattern} structure
with \emph{Google Guice} for dependency injection.

To achieve it we first had to find the parts of the code where the \emph{Cake Pattern}
was used. The \emph{ApiModule.scala} file is the class that need \emph{Dependency Injection}
of two class \emph{CookRepository} and \emph{EggRepository} defined in the
\emph{DomainModule} trait.

\begin{lstlisting}[style=scala]
trait DomainModule {
  def cookRepository: CookRepository
  def eggRepository: EggRepository
}

trait ApiModule { this: DomainModule =>
  implicit def executionContext: ExecutionContext
  val foodPrepApi: FoodPrepApi = new FoodPrepApi(cookRepository)
}
\end{lstlisting}

The actual \emph{Dependency Injection} was done in the \emph{InfrastructureModule.scala}

\begin{lstlisting}[style=scala]
trait InfrastructureModule { this: ApiModule with DomainModule =>
  override val eggRepository: EggRepository = new InMemoryEggRepository()

  override val cookRepository: CookRepository = new InMemoryCookRepository(eggRepository)

  override implicit def executionContext: ExecutionContext = scala.concurrent.ExecutionContext.global
}
\end{lstlisting}

So changing the structure of the code to use \emph{Google Guice} actually make the
\emph{DomainModule} class useless and we can therefore remove it from the application.
Instead I did my injection of the two class needed (\emph{CookRepository} and
\emph{EggRepository}) directly into the \emph{ApiModule} class.

\begin{lstlisting}[style=scala]
class ApiModule @Inject()(eggRepository: EggRepository, cookRepository: CookRepository)
\end{lstlisting}

And also \emph{Inject} into the dependencies we inject into \emph{ApiModule}.

\begin{lstlisting}[style=scala]
class InMemoryEggRepository @Inject()()(implicit ec: ExecutionContext) extends EggRepository

class InMemoryCookRepository @Inject()(eggRepository: EggRepository)(implicit ec: ExecutionContext) extends CookRepository
\end{lstlisting}

And create the bindings for \emph{Google Guice} into the \emph{InfrastructureModule}
class we completely changed.

\begin{lstlisting}[style=scala]
class InfrastructureModule extends Module {
  def configure(binder: Binder) = {
    binder.bind(classof[EggRepository]).to(classof[InMemoryEggRepository])
    binder.bind(classof[CookRepository]).to(classof[InmemoryCookRepository])
  }
}
\end{lstlisting}

So now to create an \emph{ApiModule} instance we do it this way

\begin{lstlisting}[style=scala]
trait Injector {
  Injector.inject(this)
}

object Injector {
  val injector = Guice.createInjector(new InfrastructureModule())

  def inject(obj: AnyRef) {
    injector.injectMembers(obj)
  }
}

val a: ApiModule = Injector.injector.getInstance(classOf[ApiModule])
\end{lstlisting}

The problem with this implementation is the \emph{EggRepository} instance
in \emph{ApiModule} and \emph{InMemoryCookRepository} are not the same. As we
see in the \emph{FoodPrepApiTest} test use \emph{EggRepository} directly to add
egg in the repository and then use the \emph{CookRepository} as if the eggs were
in the repository. To solve this issue I used \emph{Guice} \emph{AssistedInject}
to create a \emph{factory} for the \emph{CookRepository} dependency and choose
which \emph{EggRepository} instance to use.

\begin{lstlisting}[style=scala]
class InMemoryCookRepository @Inject()(
  @Assisted eggRepository: EggRepository
)(implicit ec: ExecutionContext) extends CookRepository

trait CookRepositoryFactory {
  def create(eggRepository: EggRepository): CookRepository
}

class InfrastructureModule extends Module {
  def configure(binder: Binder) = {
    binder.bind(classOf[EggRepository]).to(classOf[InMemoryEggRepository])
    binder.install(new FactoryModuleBuilder()
      .implement(classOf[CookRepository], classOf[InMemoryCookRepository])
      .build(classOf[CookRepositoryFactory]))
  }
}
\end{lstlisting}

And inject it in \emph{ApiModule} this way.

\begin{lstlisting}[style=scala]
class ApiModule @Inject()(eggRepo: EggRepository, cookRepo: CookRepositoryFactory) {
  val eggRepository: EggRepository = eggRepo
  val cookRepository: CookRepository = cookRepo.create(eggRepo)
}
\end{lstlisting}

Si I only had to change the way I created the \emph{ApiModule} instance in the tests
and that was all, the \emph{Dependency Injection} worked perfectly.

\bibliographystyle{apalike}

\bibliography{bib}
\end{document}
