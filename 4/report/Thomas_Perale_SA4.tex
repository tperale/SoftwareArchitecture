
\documentclass[a4paper,11pt]{article}

\usepackage[english]{babel}
\usepackage[utf8x]{inputenc}
\usepackage{amsmath}
\usepackage{graphicx}
\usepackage{cite}
\usepackage{hyperref}
\usepackage{fullpage}
\usepackage{listings}
\usepackage{xcolor}

\definecolor{dkgreen}{rgb}{0,0.6,0}
\definecolor{gray}{rgb}{0.5,0.5,0.5}
\definecolor{mauve}{rgb}{0.58,0,0.82}

\lstdefinestyle{scala}{
  frame=tb,
  language=scala,
  aboveskip=3mm,
  belowskip=3mm,
  showstringspaces=false,
  columns=flexible,
  basicstyle={\small\ttfamily},
  numbers=none,
  numberstyle=\tiny\color{gray},
  keywordstyle=\color{blue},
  commentstyle=\color{dkgreen},
  stringstyle=\color{mauve},
  frame=single,
  breaklines=true,
  breakatwhitespace=true,
  tabsize=3,
}

\setlength{\parskip}{1.2ex}
\begin{document}

\title{Software Architecture}
\title{Assignment 4: Design patterns for micro-service architectures}
\author{Thomas Perale -- 0546990}

\maketitle

\section{Introduction}

For this assignment we had to implement an \emph{Actor System} that
enable the use of filters on messages passed to the actors.

The filter actors we needed to implement was

\begin{enumerate}
  \item A license filter disregarding messages if the plate was \emph{null}.
  \item A speed filter disregarding messages if the speed was within the speed limit.
\end{enumerate}

If a \emph{message} pass through those filters it mean the \emph{vehicule} From
which originated the \emph{message} made an infraction.

To simulate the system we needed to schedule every 30 seconds a new message
with a random plate and a random speed.

\section{Ramdom message generation}

The first step of the project is generating the messages we pass through the
\emph{Actors}.

\begin{lstlisting}[style=scala]
case class PhotoMessage(licensePlate : String, speed : Int)

object PlateGenerator {
  val random = new scala.util.Random

  val MIN_SPEED = 55
  val MAX_SPEED = 95

  val PLATE_LENGTH = 6

  def generateSpeed: Int = {
    return MIN_SPEED + random.nextInt(MAX_SPEED - MIN_SPEED + 1)
  }

  def generatePlate: String = {
    return {random.alphanumeric take PLATE_LENGTH mkString}
  }

  def generate: PhotoMessage = {
    return new PhotoMessage(generatePlate, generateSpeed)
  }
}
\end{lstlisting}

We didn't have any restriction on the plate format so I choosed an arbitrary one,
consisting of 6 alphanumeric characters.

We then needed to implement the \emph{Actors} as presented in the assignment
template.

\begin{lstlisting}[style=scala]
class CameraActor(nextFilter: ActorRef) extends Actor {
  override def receive: Receive = {
    case PhotoMessage(license: String, speed: Int) =>
      //send message every 30 seconds
      nextFilter ! PhotoMessage(license, speed)
    case _ =>
      println("[error] In the message structure sended") // ERROR
  }
}

class LicenseFilterActor(nextFilter: ActorRef) extends Actor {
  override def receive: Receive = {
    case PhotoMessage("", speed: Int) =>
      // We have to halt here (exception or halt ?)
      println("[error] No plate found by the camera")

    case PhotoMessage(null, speed: Int) =>
      // We have to halt here (exception or halt ?)
      println("[error] No plate found by the camera")

    case PhotoMessage(license: String, speed: Int) =>
      // If we arrive here we have a possible infraction because the car
      // license exist so we have to continue to the next filter.
      nextFilter ! PhotoMessage(license, speed)
  }
}

class SpeedFilterActor(nextFilter: ActorRef) extends Actor {
  val errormsg: String = "Invalid message, dropped..."

  override def receive: Receive = {
    case PhotoMessage(license: String, speed: Int) => if (speed < 70) {
      // If the speed is below the limit we have to halt or throw exception
      println(s"[info] vehicule is running below the speed limit (here ${speed} km/h), disgarding")
    }
    else {
      // We proceed incase it's an infraction case.
      nextFilter ! PhotoMessage(license, speed)
    }
  }
}

class CPOfficeActor extends Actor {
  override def receive: Receive = {
    case PhotoMessage(license: String, speed: Int) =>
      // If we get till here we have an infraction
      println(s"[info] Infraction detected for car plate ${license} running at ${speed} km/h")
  }
}
\end{lstlisting}

We take care to stop the message passing when the \emph{License Plate} don't exist
or when the \emph{speed} is actually below the speed limit.

In the case the message reach the \emph{CPOfficeActor} we just print we have an
infraction but don't do anything special as nothing was told in the assignment.

Lastly to chain the different actors together and send a message every 30 seconds
we implement the \emph{assignment} object.

\begin{lstlisting}[style=scala]
object assignment extends App {
  import scala.concurrent.ExecutionContext.Implicits.global

  val system = ActorSystem("Assignment4")
  val cpoffice = system.actorOf(Props[CPOfficeActor], "cpoffice")
  val licensefilter = system.actorOf(Props(classOf[LicenseFilterActor], cpoffice), "licensefilter")
  val speedfilter = system.actorOf(Props(classOf[SpeedFilterActor], licensefilter), "speedfilter")
  val camera = system.actorOf(Props(classOf[CameraActor], speedfilter), "camera")

  system.scheduler.schedule(0 second, 30 second) {
     camera ! PlateGenerator.generate
  }
}
\end{lstlisting}

\bibliographystyle{apalike}

\bibliography{bib}
\end{document}
