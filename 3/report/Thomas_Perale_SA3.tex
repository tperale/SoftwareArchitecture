
\documentclass[a4paper,11pt]{article}

\usepackage[english]{babel}
\usepackage[utf8x]{inputenc}
\usepackage{amsmath}
\usepackage{graphicx}
\usepackage{cite}
\usepackage{hyperref}
\usepackage{fullpage}
\usepackage{listings}
\usepackage{xcolor}

\definecolor{dkgreen}{rgb}{0,0.6,0}
\definecolor{gray}{rgb}{0.5,0.5,0.5}
\definecolor{mauve}{rgb}{0.58,0,0.82}

\lstdefinestyle{scala}{
  frame=tb,
  language=scala,
  aboveskip=3mm,
  belowskip=3mm,
  showstringspaces=false,
  columns=flexible,
  basicstyle={\small\ttfamily},
  numbers=none,
  numberstyle=\tiny\color{gray},
  keywordstyle=\color{blue},
  commentstyle=\color{dkgreen},
  stringstyle=\color{mauve},
  frame=single,
  breaklines=true,
  breakatwhitespace=true,
  tabsize=3,
}

\setlength{\parskip}{1.2ex}
\begin{document}

\title{Software Architecture}
\title{Assignment 3: Using REST Connectors in SOA Architectures}
\author{Thomas Perale -- 0546990}

\maketitle

\section{Introduction}

For this assignment we had to implement two tasks:

\begin{enumerate}
  \item Process the JSON response from the iRails Connections REST API.
  \item Have form validation with the Play framework approach.
\end{enumerate}

\section{API processing}

For this section the first step of the solution was simply to read the
documentation of \emph{ScalaWS} and see how to send a \emph{GET HTTP Request} which
basically translate to this snippet of code in \emph{Scala}.

\begin{lstlisting}[style=scala]
val url = s"https://api.irail.be/connections/?from=${connectionData.from}&to=${connectionData.to}"
val futureResponse = ws.url(url).addHttpHeaders("Accept" -> "application/json").get()
\end{lstlisting}

Where \emph{connectionData} are the form datas from \emph{RouteForm.Data} object
which I will talk later.

To handle the reception of the API's datas we transform directly the response
into a \emph{Sequence} of \emph{Connection} object defined on the \emph{JsonModel.scala}
file with the following code.

\begin{lstlisting}[style=scala]
futureResponse.map { response =>
  val result: JsValue = response.json
  val c = (result \ "connection").get.as[Seq[Connection]]

  connections.appendAll(c)

  Ok(views.html.index(form, connections, postURL))
}
\end{lstlisting}

The application can automatically make the translation from \emph{JsonResult}
to \emph{Connection} object. To do it I had to define \emph{Json Readers} like
this.

\begin{lstlisting}[style=scala]
case class Connection(id: String, departure: Departure, arrival: Arrival, duration: String)
object Connection {
    implicit val readsConnection: Reads[Connection] = new Reads[Connection] {
        def reads(json: JsValue): JsResult[Connection] = {
            for {
              id <- (json \ "id").validate[String]
              departure <- (json \ "departure").validate[Departure]
              arrival <- (json \ "arrival").validate[Arrival]
              duration <- (json \ "duration").validate[String]
            } yield Connection(id, departure, arrival, duration)
        }
    }
}
case class Departure(station: String, time: String, vehicle: String, platform: String  )
object Departure {
    implicit val readsDeparture: Reads[Departure] = new Reads[Departure] {
        def reads(json: JsValue): JsResult[Departure] = {
            for {
              station <- (json \ "station").validate[String]
              time <- (json \ "time").validate[String]
              vehicule <- (json \ "vehicle").validate[String]
              platform <- (json \ "platform").validate[String]
            } yield Departure(station, time, vehicule, platform)
        }
    }
}
case class Arrival(station: String, time: String, vehicle: String, platform: String)
object Arrival {
    implicit val readsArrival: Reads[Arrival] = new Reads[Arrival] {
        def reads(json: JsValue): JsResult[Arrival] = {
            for {
              station <- (json \ "station").validate[String]
              time <- (json \ "time").validate[String]
              vehicule <- (json \ "vehicle").validate[String]
              platform <- (json \ "platform").validate[String]
            } yield Arrival(station, time, vehicule, platform)
        }
    }
}
\end{lstlisting}

And those were the first requirements for the project.

\section{Form validation}

The second requirements consisted in handling the forms datas to only accept
stations defined in the \emph{.csv} provided in the projet.

For this purpose we defined a \emph{Form} definition with customs \emph{validators}
to describe the informations the user send from the webapp.

\begin{lstlisting}[style=scala]
val validateLocation: Constraint[String] = Constraint[String]("constraint.isastation") { o =>
  if (!isStation(o)) {
    // Check if is defined in the csv provided.
    Invalid(ValidationError("error.locationnotexist"))
  } else  {
    Valid
  }
}

case class Data(from: String, to: String)

val form = Form(
  mapping(
    "From" -> text.verifying(validateLocation),
    "To" -> text.verifying(validateLocation),
  )(Data.apply)(Data.unapply)
)
\end{lstlisting}

We get the data in an \emph{Action} by calling the following method

\begin{lstlisting}[style=scala]
form.bindFromRequest.fold
\end{lstlisting}

\bibliographystyle{apalike}

\bibliography{bib}
\end{document}
