
\documentclass[a4paper,11pt]{article}

\usepackage[english]{babel}
\usepackage[utf8x]{inputenc}
\usepackage{amsmath}
\usepackage{graphicx}
\usepackage{cite}
\usepackage{hyperref}
\usepackage{fullpage}
\usepackage{listings}
\usepackage{xcolor}

\definecolor{dkgreen}{rgb}{0,0.6,0}
\definecolor{gray}{rgb}{0.5,0.5,0.5}
\definecolor{mauve}{rgb}{0.58,0,0.82}

\lstdefinestyle{scala}{
  frame=tb,
  language=scala,
  aboveskip=3mm,
  belowskip=3mm,
  showstringspaces=false,
  columns=flexible,
  basicstyle={\small\ttfamily},
  numbers=none,
  numberstyle=\tiny\color{gray},
  keywordstyle=\color{blue},
  commentstyle=\color{dkgreen},
  stringstyle=\color{mauve},
  frame=single,
  breaklines=true,
  breakatwhitespace=true,
  tabsize=3,
}

\setlength{\parskip}{1.2ex}
\begin{document}

\title{Software Architecture}
\title{Assignment 2: Using the Visitor pattern}
\author{Thomas Perale -- 0546990}

\maketitle

\section{Introduction}

For this assignment we need to extend the \emph{Cook} class of the
first assignment project to use the \emph{Visitor} design pattern in
order to support receiving a list of orders to prepare.

\section{Changing the Cook class}

We want to match the \emph{Visitor} pattern as much as possible but we also want
to keep the original structure of the program. So we have to keep in mind our ultimate
goal is to be able to pass an \emph{order list} to the \emph{Cook} class letting
it handle any type of \emph{Food} and \emph{FoodStyle}. For this purpose I made
a \emph{Order} class

\begin{lstlisting}[style=scala]
case class Order(food: RawFood, style: FoodStyle)
\end{lstlisting}

Letting you to match a new \emph{Food} type with its specific \emph{FoodStyle}.

The \emph{Cook} class no longer accept \emph{EggRepository} only but \emph{OrderRepository}
which are the sames implementations.

\begin{lstlisting}[style=scala]
class Cook(val id: CookId)(orderRepository: OrderRepository)(implicit ec: ExecutionContext) {
  import Cook._ //to get OutOfEggsException here
  private val fryingPan = new EmptyFryingPan()

  private def wait(pan: FullFryingPan) = {
    while(!pan.checkDoneness())
      Thread.sleep(10)
  }

  def prepare(): Future[CookedFood] = {
    orderRepository.findAndRemove().map { orderOption =>
      val order = orderOption.getOrElse(throw OutOfFoodsException)
      val fullFryingPan = fryingPan.add(order.food, order.style)
      wait(fullFryingPan)
      fullFryingPan.remove()._2
    }
  }
}
\end{lstlisting}

So the big change is the creation of the \emph{Food} class to not limit the foods
we prepare to eggs.

\begin{lstlisting}[style=scala]
trait Food

object Food extends Food {
  trait CookedFood {
    def style: FoodStyle
  }
  trait RawFood {
    def startCooking(style: FoodStyle): PartiallyCookedFood
  }
  trait PartiallyCookedFood extends CookedFood {
    def isDone: Boolean
    def finishCooking(): CookedFood
  }
  trait FullyCookedFood extends CookedFood
}
\end{lstlisting}

Which is implemented in specific food classes. Here is an example for the
\emph{Bacon}.

\begin{lstlisting}[style=scala]
sealed trait Bacon

object Bacon {
  sealed trait CookedBacon extends CookedFood {
    def style: BaconStyle
  }

  case class RawBacon() extends RawFood {
    def startCooking(style: FoodStyle): PartiallyCookedBacon =
      PartiallyCookedBacon(style)
  }

  case class PartiallyCookedBacon(style: FoodStyle) extends PartiallyCookedFood {
    private val startTime: Timestamp = Timestamp()
    def isDone: Boolean =
      (System.currentTimeMillis() - startTime.value) > style.cookTime.toMillis

    def finishCooking(): CookedFood =
      if(isDone) FullyCookedBacon(style) else this
  }

  case class FullyCookedBacon(style: FoodStyle) extends FullyCookedFood
}
\end{lstlisting}

I do something similar for the \emph{Style} class to define specific \emph{FoodStyle}
for specific \emph{Food}

\begin{lstlisting}[style=scala]
trait FoodStyle {
  def cookTime = 20.millis
}

sealed trait BaconStyle extends FoodStyle
object BaconStyle {
  case object American extends BaconStyle
  case object Candied extends BaconStyle
  case object Applewood extends BaconStyle
}
\end{lstlisting}

Order are prepared one at a time everytime you call \emph{.prepare()} on a
you prepare an order. We can see the test to see how to use the new implementation
in more details.

\begin{lstlisting}[style=scala]
orderRepository.add(createOrder(createRawBacon(), BaconStyle.American))
orderRepository.add(createOrder(createRawWaffle(), WaffleStyle.Brussels))
orderRepository.add(createOrder(createRawBacon(), BaconStyle.American))

val expectedBacon = createCookedBacon(BaconStyle.American)
val expectedWaffle = createCookedWaffle(WaffleStyle.Brussels)

whenReady(foodPrepApi.prepare()) { item =>
  assert(item == expectedBacon)
}
\end{lstlisting}

\bibliographystyle{apalike}

\bibliography{bib}
\end{document}
